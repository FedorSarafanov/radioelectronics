%!TEX root = ../problems.tex

\begin{task}
Определить отклик выхода RL-цепи, изображенной на рисунке, на воздействие прямоугольного импульса длительностью $\tau_0$. Нарисовать график $u_\out\qty(t)$. При выполнении какого условия будет осуществляться приближенное интегрирование входной цепи? 
\end{task}

\begin{proof}[\rm{\textbf{Решение}}]
Эквивалентная схема (картиночка)
 \begin{equation}
	u_\in\qty(t)\risingdotseq \frac{A}{p}-\frac{A}{p}\cdot e^{-p\tau}=\frac{A}{p}\qty(1-e^{-p\tau})
\end{equation}
По второму правилу Кирхгофа, сумма падений напряжения на всех элементах цепи равна ЭДС. В нашем случае возможное начальное напряжение на катушке мы относим к ЭДС, а сумму падений напряжения записываем как ток в контуре на суммарный импеданс контура:
\begin{equation}
	\eds=Z(p)\cdot I(p) \quad \Rightarrow \quad
	\frac{A}{p} \qty(1-e^{-p\tau}) + i_L\qty(0)L=\qty(pL+R)I\qty(p)
\end{equation}
Отсюда выражаем ток в контуре:
\begin{equation}
	I\qty(p)=\frac{\frac{A}{p}\qty(1-e^{-p\tau})}{pL+R} + 
		\frac{i_L\qty(0)L}{pL+R}
\end{equation}
% С другой стороны, $u_\in=u_R+u_L$, и отсюда
% \begin{gather}
% \frac{A}{p}\qty(1-e^{-p\tau})=
% 	I\qty(p)R+u_L\qty(p) \Rightarrow \\
% %
% u_L\qty(p) = 
% 	\frac{A}{p}\qty(1-e^{-p\tau})-I\qty(p)R =
% 	\frac{A}{p}\qty(1-e^{-p\tau}) - 
% 		\frac{\frac{AR}{p}\qty(1-e^{-p\tau})}{pL+R} - 
% 		\frac{i_L\qty(0)LR}{pL+R} = \\ =
% %
% \frac{A}{p}\qty(1-e^{-p\tau}) - 
% 	\frac{A\frac{R}{L}}{p\qty(p+\frac{R}{L})} + 
% 	\frac{A\frac{R}{L}}{p\qty(p+\frac{R}{L})}e^{-p\tau} - 
% 	\frac{i_L\qty(0)R}{p+\frac{R}{L}} \LT \\ \LT
% %  
% \cancel{A\H\qty(t)}-
% 	\cancel{A\H\qty({t}-\tau)}-
% 	A\qty(\cancel{1}-e^{-\frac{Rt}{L}}) \H\qty(t) +
% 	A\qty(\cancel{1}-e^{-\frac{R}{L}\qty(t-\tau)}) \H\qty(t-\tau) - 
% 	i_L\qty(0)Re^{\frac{-R}{L}t} \H\qty(t) = \\=
% %
% Ae^{-\frac{Rt}{L}}\cdot \H\qty(t)-
% 	Ae^{-\frac{R\qty(t-\tau)}{L}}\cdot \H\qty(t-\tau)-
% 	i_L\qty(0)Re^{-\frac{Rt}{L}}\cdot \H\qty(t)
% \end{gather}
% Отсюда получаем окончательно выражение для напряжения на индуктивности:
% \begin{equation}
% u_L\qty(t)=
% \qty(A-i_L\qty(0)R)e^{-\frac{Rt}{L}}\cdot \H\qty(t) - 
% 	Ae^{-\frac{R\qty(t-\tau)}{L}}\cdot \H\qty(t-\tau)
% \end{equation}
% Нам надо
Теперь мы можем найти и выходное напряжение -- напряжение на резисторе:
\begin{gather}
	u_\out (p) \equiv u_R\qty(p)=I\qty(p)R=
		\frac{AR\qty(1-e^{-p\tau})}{p\qty(pL+R)}+
		\frac{i_L\qty(0)RL}{pL+R}=
	\frac{\frac{AR}{L}\qty(1-e^{-p\tau})}{p\qty(p+\frac{R}{L})}+
		\frac{i_L\qty(0)R}{p+\frac{R}{L}} 
		=\\=
	A\frac{\frac{R}{L}}{p\qty(p+\frac{R}{L})}-
		A\frac{\frac{R}{L}}{p\qty(p+\frac{R}{L})}e^{-p\tau}+
		i_L\qty(0)R\frac{1}{p+\frac{R}{L}} 
		% \LT \\ \LT 
	% A\qty(1-e^{-\frac{Rt}{L}})\cdot \H\qty(t)-
		% A\qty(1-e^{-\frac{R\qty(t-\tau)}{L}})\cdot \H\qty(t-\tau)+
		% i_L\qty(0)Re^{-\frac{Rt}{L}}\cdot \H\qty(t)
\end{gather}
Используя свойства преобразования Лапласа
\begin{gather}
	\frac{\alpha}{p(p+\alpha)}\LT (1-e^{-\alpha t})\H(t)\\
	\frac{1}{(p+\alpha)}\LT e^{-\alpha t}\H(t)\\
	e^{{-p\tau}}F(p) \LT f(t-\tau)\H(t-\tau), \qq{где} F(p) \LT f(t)
\end{gather}
Из выражения $u_\out(p)$ элементарно получаем оригинал $u_\out(t)$:
\begin{equation}
	u_\out(t)=A\qty(1-e^{-\frac{Rt}{L}})\cdot \H\qty(t)-
		A\qty(1-e^{-\frac{R\qty(t-\tau)}{L}})\cdot \H\qty(t-\tau)+
		i_L\qty(0)Re^{-\frac{Rt}{L}}\cdot \H\qty(t)
\end{equation}
\paragraph{Условие интегрирования.} 
\begin{gather}
	% u_\out\qty(t)=A\qty(1-e^{-\frac{Rt}{L}})\cdot \H\qty(t)-A\qty(1-e^{-\frac{R\qty(t-\tau)}{L}})\cdot \H\qty(t-\tau)+i_L\qty(0)Re^{-\frac{Rt}{L}}\cdot \H\qty(t) 
% \\\tau?>>\tau |\frac{L}{R}>>\tau_| 
\end{gather}
\end{proof}