\begin{task}
Определить отклик выхода RL-цепи, изображенной на рисунке, на воздействие прямоугольного импульса длительностью $\tau_0$. Нарисовать график $u_\out(t)$. При выполнении какого условия будет осуществляться приближенное интегрирование входной цепи? 
\end{task}

\begin{proof}[\rm{\textbf{Решение}}]
Эквивалентная схема (картиночка)
 \begin{equation}
	u_\in(t)\risingdotseq \frac{A}{p}-\frac{A}{p}\cdot e^{-p\tau}=\frac{A}{p}(1-e^{-p\tau})
\end{equation}
По второму правилу Кирхгофа:
\begin{equation}
	\frac{A}{p} (1-e^{-p\tau}) + i_L(0)L=(pL+R)I(p)
\end{equation}
\begin{equation}
	I(p)=\frac{\frac{A}{p}(1-e^{-p\tau})}{pL+R} + 
		\frac{i_L(0)L}{pL+R}
\end{equation}
С другой стороны,
\begin{gather}
\frac{A}{p}(1-e^{-p\tau})=
	I(p)R+u_L(p) \Rightarrow \\
%
u_L(p) = 
	\frac{A}{p}(1-e^{-p\tau})-I(p)R =
	\frac{A}{p}(1-e^{-p\tau}) - 
		\frac{\frac{AR}{p}(1-e^{-p\tau}}{pL+R} - 
		\frac{i_L(0)LR}{pL+R} = \\ =
%
\frac{A}{p}(1-e^{-p\tau}) - 
	\frac{A\frac{R}{L}}{p(p+\frac{R}{L}} + 
	\frac{A\frac{R}{L}}{p(p+\frac{R}{L})}e^{-p\tau} - 
	\frac{i_L(0)R}{p+\frac{R}{L}} \LT \\ \LT
%  
\cancel{A\H(t)}-
	\cancel{A\H({t}-\tau)}-
	A(\cancel{1}-e^{-\frac{Rt}{L}}) \H(t) +
	A(\cancel{1}-e^{-\frac{R}{L}(t-\tau)}) \H(t-\tau) - 
	i_L(0)Re^{\frac{-R}{L}t} \H(t) = \\=
%
Ae^{-\frac{Rt}{L}}\cdot \H(t)-
	Ae^{-\frac{R(t-\tau)}{L}}\cdot \H(t-\tau)-
	i_L(0)Re^{-\frac{Rt}{L}}\cdot \H(t)
\end{gather}
Итак,
\begin{equation}
u_L(t)=
(A-i_L(0)R)e^{-\frac{Rt}{L}}\cdot \H(t) - 
	Ae^{-\frac{R(t-\tau)}{L}}\cdot \H(t-\tau)
\end{equation}
Нам надо
\begin{gather}
	u_R(p)=I(p)R=
		\frac{AR(1-e^{-p\tau})}{p(pL+R)}+
		\frac{i_L(0)RL}{pL+R}=
	\frac{\frac{AR}{L}(1-e^{-p\tau})}{p(p+\frac{R}{L}}+
		\frac{i_L(0)R}{p+\frac{R}{L}}\LT \\ \LT 
	A(1-e^{-\frac{Rt}{L}})\cdot \H(t)-
		A(1-e^{-\frac{R(t-\tau)}{L}})\cdot \H(t-\tau)+
		i_L(0)Re^{-\frac{Rt}{L}}\cdot \H(t)
\end{gather}
\paragraph{Условие интегрирования.} 
\begin{gather}
	u_\out(t)=A(1-e^{-\frac{Rt}{L}})\cdot \H(t)-A(1-e^{-\frac{R(t-\tau)}{L}})\cdot \H(t-\tau)+i_L(0)Re^{-\frac{Rt}{L}}\cdot \H(t) 
\\\tau?>>\tau |\frac{L}{R}>>\tau_| 
\end{gather}
\end{proof}