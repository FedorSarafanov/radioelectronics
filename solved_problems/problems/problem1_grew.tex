\begin{task}
	Найти спектр сигнала $S(t)=A\cdot e^{-\frac{t^2}{\tau^2}}$. Нарисовать график $|S(\omega)|$. Что будет при разных $\tau$?
\end{task}
\begin{proof}[\rm{\textbf{Решение}}]
	\^S$(\omega)=A\int\limits_{-\infty}^{+\infty}e^{-\frac{t^2}{\tau^2}}e^{-i\omega t}dt=A\int\limits_{-\infty}^{+\infty}e^{-\qty(\frac{t^2}{\tau^2}+i\omega t)}dt$

	Выделим полный квадрат в степени экспоненты:
	\begin{equation}
		\text{\^S}(\omega) = A\int\limits_{-\infty}^{+\infty}e^{-\qty(\frac{t}{\tau}+\frac{i\omega \tau}{2})^2-\frac{\omega^2 \tau^2}{4}}dt
	\end{equation}
	Сделаем замену переменных: $\frac{t}{\tau}+\frac{i\omega \tau}{2} = x$ , $t=\tau x-\frac{i\omega \tau}{2}\tau$ , $dt=\tau dx$
	\begin{equation}
		\text{\^S}(\omega)=A\tau e^{\frac{-\omega^2 \tau^2}{4}} \int\limits_{-\infty}^{+\infty}e^{-x^2}dx
	\end{equation}
	Это интеграл Пуассона, тогда: 
	\begin{equation}
		\text{\^S}(\omega)=A\tau e^{\frac{-\omega^2 \tau^2}{4}} \frac{\sqrt{\pi}}{2}
	\end{equation}

	Чем медленнее изменяется U(t) (т.е. чем больше $\tau$ ), тем быстрее изменяется $|S(\omega)|$ (т.е. тем уже спектр сигнала), и наоборот.
	Похоже ли это на интеграл Пуассона? Наверное, не должно быть деления на 2
\end{proof}