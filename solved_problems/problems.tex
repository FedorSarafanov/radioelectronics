% Тип документа
\documentclass[a4paper,14pt]{extarticle}

% Шрифты, кодировки, символьные таблицы, переносы
\usepackage{cmap}
\usepackage[T2A]{fontenc}
\usepackage[utf8x]{inputenc}
\usepackage[russian]{babel}

\usepackage
	{
		% Дополнения Американского математического общества (AMS)
		amssymb,
		amsfonts,
		amsmath,
		amsthm,
		physics,
		% Графики и рисунки
		graphicx,
		color,
		geometry,
		multicol,
		tikz,
		hyperref
	}  


% Увеличенный межстрочный интервал, французские пробелы
\linespread{1.3} 
\frenchspacing 


\geometry		
	{
		left			=	1cm,
		right 			=	1cm,
		top 			=	2cm,
		bottom 			=	2cm,
		bindingoffset	=	0cm
	}
\usepackage{fancyhdr}
\pagestyle{fancy}
\fancyhf{}
\rhead{Сарафанов Ф.Г.}
\chead{Экзамен по радиоэлектронике, зима 2019}
\lhead{Типовые задачи}
\cfoot{\thepage}

%%%%%%%%%%%%%%%%%%%%%%%%%%%%%%%%%%%%%%%%%%%%%%%%%%%%%%%%%%%%%%%%%%%%%%%%%%%%%%%
\gdef\rot{\operatorname{rot}} \gdef\div{\operatorname{div}}
% \gdef\E{\vec{E}}
% \gdef\D{\vec{D}}
% \gdef\H{\vec{H}}
% \gdef\B{\vec{B}}
% \gdef\j{\vec{j}}
% \gdef\n{\vec{n}}

\gdef\LT{\risingdotseq}
\gdef\out{\text{вых}}
\gdef\in{\text{вх}}
\gdef\newint{\int\limits_{-\infty}^{+\infty}}

\usepackage{lipsum}  
\usepackage{ifthen}
\newcommand{\frc}[2]{\raisebox{2pt}{$#1$}\big/\raisebox{-3pt}{$#2$}}
\usepackage{mathtools}  
\mathtoolsset{showonlyrefs}  
% \usepackage[export]{adjustbox}
\usepackage{cancel}
\DeclareMathAlphabet{\mymathbb}{U}{BOONDOX-ds}{m}{n}
\gdef\H{\mymathbb{1}}

\theoremstyle{definition}
\newtheorem*{task}{Дано}

% №№№№№№№№№№№№№№№№№№№№№№№№№№№№№№№№№№№№№№№№№№№№№№№№№№№№№№№№№№№№№№№№№№№№№№№№№№№№№№№№№№№№№№№№№№№№№№№№№№№№№№№№№№№№
% №№№№№№№№№№№№№№№№№№№№№№№№№№№№№№№№№№№№№№№№№№№№№№№№№№№№№№№№№№№№№№№№№№№№№№№№№№№№№№№№№№№№№№№№№№№№№№№№№№№№№№№№№№№№
\begin{document}
% №№№№№№№№№№№№№№№№№№№№№№№№№№№№№№№№№№№№№№№№№№№№№№№№№№№№№№№№№№№№№№№№№№№№№№№№№№№№№№№№№№№№№№№№№№№№№№№№№№№№№№№№№№№№
% №№№№№№№№№№№№№№№№№№№№№№№№№№№№№№№№№№№№№№№№№№№№№№№№№№№№№№№№№№№№№№№№№№№№№№№№№№№№№№№№№№№№№№№№№№№№№№№№№№№№№№№№№№№№


\section{Типовые задачи по радиоэлектронике}

\subsection{Задача №1}

\begin{task}
	Найти спектр сигнала $S(t)=A\cdot e^{-\frac{t^2}{\tau^2}}$. Нарисовать график $|S(\omega)|$. Что будет при разных $\tau$?
\end{task}
\begin{proof}[\rm{\textbf{Решение}}]
	\^S$(\omega)=A\int\limits_{-\infty}^{+\infty}e^{-\frac{t^2}{\tau^2}}e^{-i\omega t}dt=A\int\limits_{-\infty}^{+\infty}e^{-\qty(\frac{t^2}{\tau^2}+i\omega t)}dt$

	Выделим полный квадрат в степени экспоненты:
	\begin{equation}
		\text{\^S}(\omega) = A\int\limits_{-\infty}^{+\infty}e^{-\qty(\frac{t}{\tau}+\frac{i\omega \tau}{2})^2-\frac{\omega^2 \tau^2}{4}}dt
	\end{equation}
	Сделаем замену переменных: $\frac{t}{\tau}+\frac{i\omega \tau}{2} = x$ , $t=\tau x-\frac{i\omega \tau}{2}\tau$ , $dt=\tau dx$
	\begin{equation}
		\text{\^S}(\omega)=A\tau e^{\frac{-\omega^2 \tau^2}{4}} \int\limits_{-\infty}^{+\infty}e^{-x^2}dx
	\end{equation}
	Это интеграл Пуассона, тогда: 
	\begin{equation}
		\text{\^S}(\omega)=A\tau e^{\frac{-\omega^2 \tau^2}{4}} \frac{\sqrt{\pi}}{2}
	\end{equation}

	Чем медленнее изменяется U(t) (т.е. чем больше $\tau$ ), тем быстрее изменяется $|S(\omega)|$ (т.е. тем уже спектр сигнала), и наоборот.
	Похоже ли это на интеграл Пуассона? Наверное, не должно быть деления на 2
\end{proof}







\subsection{Задача №2}
\begin{task}
Определить отклик выхода RL-цепи, изображенной на рисунке, на воздействие прямоугольного импульса длительностью $\tau_0$. Нарисовать график $u_\out(t)$. При выполнении какого условия будет осуществляться приближенное интегрирование входной цепи? 
\end{task}

\begin{proof}[\rm{\textbf{Решение}}]
Эквивалентная схема (картиночка)
 \begin{equation}
	u_\in(t)\risingdotseq \frac{A}{p}-\frac{A}{p}\cdot e^{-p\tau}=\frac{A}{p}(1-e^{-p\tau})
\end{equation}
По второму правилу Кирхгофа:
\begin{equation}
	\frac{A}{p} (1-e^{-p\tau}) + i_L(0)L=(pL+R)I(p)
\end{equation}
\begin{equation}
	I(p)=\frac{\frac{A}{p}(1-e^{-p\tau})}{pL+R} + 
		\frac{i_L(0)L}{pL+R}
\end{equation}
С другой стороны,
\begin{gather}
\frac{A}{p}(1-e^{-p\tau})=
	I(p)R+u_L(p) \Rightarrow \\
%
u_L(p) = 
	\frac{A}{p}(1-e^{-p\tau})-I(p)R =
	\frac{A}{p}(1-e^{-p\tau}) - 
		\frac{\frac{AR}{p}(1-e^{-p\tau}}{pL+R} - 
		\frac{i_L(0)LR}{pL+R} = \\ =
%
\frac{A}{p}(1-e^{-p\tau}) - 
	\frac{A\frac{R}{L}}{p(p+\frac{R}{L}} + 
	\frac{A\frac{R}{L}}{p(p+\frac{R}{L})}e^{-p\tau} - 
	\frac{i_L(0)R}{p+\frac{R}{L}} \LT \\ \LT
%  
\cancel{A\H(t)}-
	\cancel{A\H({t}-\tau)}-
	A(\cancel{1}-e^{-\frac{Rt}{L}}) \H(t) +
	A(\cancel{1}-e^{-\frac{R}{L}(t-\tau)}) \H(t-\tau) - 
	i_L(0)Re^{\frac{-R}{L}t} \H(t) = \\=
%
Ae^{-\frac{Rt}{L}}\cdot \H(t)-
	Ae^{-\frac{R(t-\tau)}{L}}\cdot \H(t-\tau)-
	i_L(0)Re^{-\frac{Rt}{L}}\cdot \H(t)
\end{gather}
Итак,
\begin{equation}
u_L(t)=
(A-i_L(0)R)e^{-\frac{Rt}{L}}\cdot \H(t) - 
	Ae^{-\frac{R(t-\tau)}{L}}\cdot \H(t-\tau)
\end{equation}
Нам надо
\begin{gather}
	u_R(p)=I(p)R=
		\frac{AR(1-e^{-p\tau})}{p(pL+R)}+
		\frac{i_L(0)RL}{pL+R}=
	\frac{\frac{AR}{L}(1-e^{-p\tau})}{p(p+\frac{R}{L}}+
		\frac{i_L(0)R}{p+\frac{R}{L}}\LT \\ \LT 
	A(1-e^{-\frac{Rt}{L}})\cdot \H(t)-
		A(1-e^{-\frac{R(t-\tau)}{L}})\cdot \H(t-\tau)+
		i_L(0)Re^{-\frac{Rt}{L}}\cdot \H(t)
\end{gather}
\paragraph{Условие интегрирования.} 
\begin{gather}
	u_\out(t)=A(1-e^{-\frac{Rt}{L}})\cdot \H(t)-A(1-e^{-\frac{R(t-\tau)}{L}})\cdot \H(t-\tau)+i_L(0)Re^{-\frac{Rt}{L}}\cdot \H(t) 
\\\tau?>>\tau |\frac{L}{R}>>\tau_| 
\end{gather}
\end{proof}





\subsection{Задача №3}
\begin{task}
	Определите отклик $u_\out(t)$ $RL$-цепи, изображенной на рисунке, на воздействие единичного импульса длительностью $\tau$. 
	Нарисуйте график отклика. 
	Какова переходная характеристика цепи? 
	При выполнении какого условия будет осуществляться приближённое дифференцирование входной цепи?
	Решить задачу с ненулевыми начальными условиями.
\end{task}
\begin{proof}[\rm{\textbf{Решение}}]
Найдем образ входного импульса преобразованием Лапласа: 
\begin{equation}
	u_\in(t)=E\cdot\H(t)-E\cdot\H(t-\tau)
	\quad\Rightarrow\quad
	u_\in(t)\LT \frac{E}{p}-\frac{E}{p}e^{-p\tau}=
	\frac{E}{p}\qty(1-e^{-p\tau})
\end{equation}
Надо учесть, что в контуре могут быть заданы начальные условия - ток $i_0$.
Тогда начальное напряжение на катушке $u_L(0)=i_0\cdot pL$, а его образ $u_L(0) \LT \frac{i_0 pL}{p}=i_0L$. 
Это напряжение можно трактовать как часть ЭДС.

Обозначим суммарный ток в контуре за $I(p)$. Тогда, так как сумма падений напряжения на каждом элементе равна нулю, получим следующее выражение:
\begin{equation}
	\frac{E}{p}\qty(1-e^{-p\tau})+i_0L=(R+pL)I(p)
\end{equation}
Откуда выразим ток $I$:
\begin{equation}
	I(p)=\frac{\frac{E}{p}\qty(1-e^{-p\tau})+i_0L}{R+pL}
\end{equation}
С другой стороны, $u_\in=u_C+u_R$, а $u_C\equiv u_\out$, тогда
\begin{gather}
	u_\out(p)=u_\in(p)-u_R(p)=u_\in(p)-I(p)R=\\=
	u_\in(p)-\frac{E(1-e^{-p\tau})R}{p(R+pL)}+\frac{i_0LR}{R+pL}=
	u_\in(p)-\frac{ER}{p(R+pL)}+\frac{ERe^{-p\tau}}{p(R+pL)}-\frac{i_0LR}{R+pL}=\\=
	u_\in(p)-\frac{E\frac{R}{L}}{p(p+\frac{R}{L})}+\frac{E\frac{R}{L}e^{-p\tau}}{p(p+\frac{R}{L})}-\frac{i_0R}{p+\frac{R}{L}}
\end{gather}
Используем свойства преобразования Лапласа:
\begin{gather}
	\frac{\alpha}{p(p+\alpha)}\LT (1-e^{-\alpha t})\H(t)\\
	\frac{1}{(p+\alpha)}\LT e^{-\alpha t}\H(t)\\
	e^{{-p\tau}}F(p) \LT f(t-\tau)\H(t-\tau), \qq{где} F(p) \LT f(t)
\end{gather}
Учтя, что $u_\in(p) \LT u_\in(t)$, произведем преобразование:
\begin{gather}
	u_\out(t)=u_\in(t)-E(1-e^{-\frac{R}{L} t})\H(t)+E(1-e^{-\frac{R}{L} (t-\tau)})\H(t-\tau)-i_0Re^{-\frac{R}{L} t}\H(t)=\\=
	\cancel{E\cdot\H(t)}-\cancel{E\cdot\H(t-\tau)}-E(\cancel{1}-e^{-\frac{R}{L} t})\H(t)+E(\cancel{1}-e^{-\frac{R}{L} (t-\tau)})\H(t-\tau)-i_0Re^{-\frac{R}{L} t}\H(t)=\\=\qty(E-i_0R)e^{-\frac{R}{L} t}\H(t)-Ee^{-\frac{R}{L}(t-\tau)}\H(t-\tau)
\end{gather}
Окончательно получили ответ: при воздействии прямоугольным импульсом $u_\in(t)$ амплитуды $E$ и длительностью $\tau$, на выходе получаем
\begin{equation}
 	u_\out(t)= \qty(E-i_0R)e^{-\frac{R}{L} t}\H(t)-Ee^{-\frac{R}{L}(t-\tau)}\H(t-\tau)
\end{equation} 
\paragraph{Условие дифференцирования.} 
Как нетрудно догадаться,
\begin{equation}
	u_\in=u_L+u_R=L\dv{I}{t}+IR
\end{equation}
Продифференцируем это выражение:
\begin{equation}
	\dv{u_\in}{t}=\underbrace{L\dv[2]{I}{t}}_{L\dv{u_L}{t}}+\frac{R}{L}\underbrace{L\dv{I}{t}}_{u_L\equiv u_\out}
\end{equation}
Если будет выполнено условие
\begin{equation}
	\abs{L\dv{u_L}{t}} \ll \abs{\frac{R}{L} u_L}
\end{equation}
Тогда будет видно, что цепочка осуществляет дифференцирование:
\begin{equation}
	u_\out=\frac{L}{R}\dv{u_\in}{t}
\end{equation}
\end{proof}

\end{document}
