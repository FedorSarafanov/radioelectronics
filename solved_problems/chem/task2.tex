\documentclass[border=1pt]{standalone}
\usepackage[europeanresistors,americaninductors]{circuitikz}
\usepackage{cmap}
\usepackage[T2A]{fontenc}
\usepackage[utf8x]{inputenc}
\usepackage[russian]{babel}
\usepackage
  {
    % Дополнения Американского математического общества (AMS)
    amssymb,
    amsfonts,
    amsmath
  }
\tikzset{
  pics/carc/.style args={#1:#2:#3}{
    code={
      \draw[pic actions] (#1:#3) arc(#1:#2:#3);
    }
  }
}

\begin{document}
	
      \begin{circuitikz}[scale=0.5]
      \draw (0,0) to [short,o-,L, l^={$L$}](4,0);

      \draw (0,-2) node  {$u_\text{вх}$};
      \draw (7,-2) node  {$u_\text{вых}$};

      \draw[thick] (2,-2) pic[red, latex-]{carc=-150:150:.5cm} node {$I$};
      % \draw[thick] (6,-2) pic[red, -latex]{carc=-150:150:1.3cm} node {$J_2$};
      % \draw[thick] (10,-2) pic[red, -latex]{carc=-150:150:1.3cm} node {$J_3$};

      \draw (0,-4) to [short,o-](4,-4);
      % to  [short,*-](8,-4)
      % to  [short,*-](12,-4)
      % to  [short,-o] (14,-4);

      \draw (4,0) to [R=$R$](4,-4);
      % \draw (8,0) to [R=$R$](8,-4);
      \draw (4,0) to [short,-o](7,0);
      \draw (4,-4) to [short,-o](7,-4);

      \end{circuitikz}
\end{document}